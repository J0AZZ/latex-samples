\documentclass[11pt]{article}
\usepackage{blindtext}
\usepackage[utf8]{inputenc}
\usepackage{ragged2e}
\usepackage{amssymb}
\usepackage[a4paper]{geometry}
\usepackage{fullpage}

\justify

% \title{Atividade 1 - MDTC}
% \author{joas.brito@academico.ufpb.br }
% \date{March 2021}

\renewcommand*\contentsname{Sumário}
	
\begin{document}

\begin{flushleft}
\tableofcontents
\end{flushleft}

% \maketitle

\justify

\section{Variável Aleatória Contínua} Uma variável $\mathnormal{X}$ é denominada de \textbf{variável aleatória contínua (v.a.c)} quando seu espaço amostral $\mathnormal{ R_x }$ é um conjunto infinito não enumerável. 
Como exemplos de variáveis aleatórias contínuas podemos citar:

\begin{itemize}
    \item resistência de um material,
    \item concentração de CO2 na água,
    \item tempo de vida de um componente eletrônico,
    \item tempo de resposta de um sistema computacional,
    \item temperatura e
    \item medições (peso, altura, comprimento,...).
\end{itemize}

\subsection{Função de Densidade}
Seja $\mathnormal{X}$ uma variável aleatória contínua (v.a.c). A função $\mathnormal{f(x)}$ que associa a cada $\mathnormal{x \in R_X}$ um número real que satisfaz as seguintes condições:
\begin{enumerate}
    \item $ f(x) \geqslant 0 $, para todo $ x \in R_X $ e
    \item $\displaystyle\int_{-\infty}^{\infty} f(x) dx = 1$,
\end{enumerate}
é denominada de \textbf{função densidade de probabilidade (fdp)} da variável aleatória $X$.

Neste caso $f(x)$ representa apenas a densidade no ponto $x$, ao contrário da função de probabilidade $p(x)$ de uma variável aleatória discreta, $f(x)$ aqui \textbf{não é a probabilidade} da variável $X$ assumir o valor $x$. Veremos a seguir como se calcula probabilidades quando se tem uma distribuição contínua.

\subsection{Cálculo das Probabilidades}
Seja $X$ uma v.a.c. com função densidade de probabilidade $f(x)$. Sejam $a < b$, dois números reais. Define-se:
\begin{center}
    $P(a < X < b) = \displaystyle\int_a^b f(x) dx$,
\end{center}
isto é, a probabilidade de que $X$ assuma valores entre os números "a" e "b" é a área sob o gráfico da função $f(x)$ entre os pontos $x=a$ e $x=b$. \\ \\ \\


\end{document}
